\documentclass[12pt,a4paper]{article}

\usepackage{Dayproblem}

\begin{document}

\begin{task}{1}
    Орбитальная скорость Земли внезапно увеличилась в $\sqrt{2}$ раз и планета отправилась осваивать новое космическое пространство. Во сколько раз Земля будет быстрее Марса в момент пересечения его орбиты?
\end{task}

\begin{solution}
    Так как скорость увеличилась в $\sqrt{2}$ раза, Земля вышла на параболическую орбиту, для скорости на которой для расстояния $r$ выполняется простое соотношение:
    $$
    v_{\Earth}(r) = \sqrt{\frac{2GM}{r}}.
    $$
    В то же время Марс двигается по круговой орбите, для которой скорость константная и равна
    $$
    v_{\mars}(r) = \sqrt{\frac{2GM}{r}}.
    $$
    Тогда отношение скоростей Земли и Марса будет равно отношению скорости на круговой орбите к скорости на параболической орбите, то есть $\sqrt{2}$
\end{solution}

\answer{$\sqrt{2}$} % ответ на задачу

\end{document}